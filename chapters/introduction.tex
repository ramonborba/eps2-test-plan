%
% introduction.tex
%
% Copyright (C) 2024 by SpaceLab.
%
% EPS 2.0 Qualification Test Plan
%
% This work is licensed under the Creative Commons Attribution-ShareAlike 4.0
% International License. To view a copy of this license,
% visit http://creativecommons.org/licenses/by-sa/4.0/.
%

%
% \brief Introduction chapter.
%
% \author Ramon de Araujo Borba <ramonborba97@gmail.com>
%
% \version 0.1.0
%
% \date 2024/02/14
%

\chapter{Introduction} \label{ch:introduction}

This document presents a test plan for the qualification of the design of the EPS 2.0 module, developed for the FloripaSat-2 service platform, according to the requirements defined for the GOLDS-UFSC satellite.

The development of this document follow the guidelines and recommendations outlined in the EPS Test Plan Guidelines \cite{eps-guidelines} document, which is based on the ECSS-E-ST-10-03 \cite{ecss-e-st-10-03} standard for testing.

The test plan was prepared considering SpaceLab's laboratory and equipment as the test facility, and tests were selected based on that premise.
For this reason, environmental tests were not included.

Multiple EPS designs are being developed at SpaceLab and there is an interest in performing performance comparisons between the different modules.
With that in mind, this test plan will also include tests aimed at fulfilling the purpose of providing data for future performance comparisons.

\section{Objectives}

The objectives of this test plan are:

\begin{itemize}
    \item The qualification of the design of the EPS 2.0 module, ensuring that the design of the module is capable of performing in accordance with its specifications.
    
    \item To generate data regarding the performance of this module for later comparison with different designs developed at SpaceLab.
\end{itemize}